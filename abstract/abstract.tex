\chapter*{Abstract}
\addcontentsline{toc}{chapter}{Abstract} % Ajouter à la table des matières.
\setlength{\parindent}{0cm}
\setlength{\parskip}{1em}
Several organizations such as the International Energy Agency (IEA) and the
Intergovernmental Panel on Climate Change (IPCC) confirmed that global warming
is undeniable and originates from human activity.
Following the 4DS IEA scenario, global energy demand is set to grow by
\SI{37}{\percent} by \num{2040}.
While policy choices and market developments bring the share of fossil fuels in
primary energy demand down to just under three-quarters
by 2040, they are not enough to stem the rise in energy-related \ce{CO2}
emissions, which grow by one-fifth. This puts the world on a path consistent
with a long-term global average temperature increase of
\SI{3.6}{\degreeCelsius}.
The IPCC estimates that in order to limit this temperature increase to
\SI{2}{\degreeCelsius} – the internationally agreed goal to avert the most
severe and widespread implications of climate change – the world cannot emit
more than around \num{1000} gigatonnes of \ce{CO2} from \num{2014} onwards.
According to IEA and IPCC, carbon capture and storage (CCS) technology could
reduce \ce{CO2} emissions by \SI{20}{\percent} of \ce{CO2}.
This decade is critical for moving CCS through and beyond the demonstration
phase. This means that urgent action is required, beginning now, from industry
and governments to develop technology and the required business models, and to
implement
incentive
frameworks that can help drive CCS deployment in the power sector and industrial
applications.
If CCS is to have a positive environmental impact then the injected \ce{CO2}
must be stored in geological reservoirs allowing to accommodate huge volumes of
\ce{CO2} for as long as it takes for anthropogenic output rates to drop to
acceptable levels and for the carbon cycle to have recovered and stabilized in
geological reservoirs. This constraint requires \ce{CO2} to be stored for
timescales of the order of \num{e4} or even \num{e4} years. To meet this
requirement we must ensure that it is not possible for injected \ce{CO2} to
migrate on large distances either vertically or horizontally away from the targeted
reservoir. \par

On the basis of these observations, this thesis proposes a workflow for the
timelapse seismic monitoring and the uncertainty assessment of the
\ce{CO2} injection suited to the environment in which porosities and
permeabilities are very low such as the St. Lawrence Lowlands (Québec, Canada) context.
This two-pronged approach use first laboratory measurements and vertical
seismic profiling as high resolution tool in order to assess the seismic
response generated by \ce{CO2} injections. Then, a logical sequence of
stochastic modeling of a potential reservoir for \ce{CO2} sequestration is
defined.
Laboratory measurements on two geological samples from the reservoir units
of the St Lawrence Lowlands has allowed to assess the seismic
response under various temperature and pressions conditions. The results
obtained have helped to calibrate the geological model employed in the seismic
modeling step. The results of the seismic modeling showed that the seismic
signature of the \ce{CO2} is mainly observable by a delay of
\SI{30}{\milli\second} related to a decrease in the wave velocities when
supercritical \ce{CO2} replace brine in the pore space.\\
Numerical experiments based on a realistic heterogeneous saline aquifer model
indicates that, given initial static data, the inversion approach should allow
for faithful properties estimation and reliable prediction of the spatial
distribution of \ce{CO2}.
