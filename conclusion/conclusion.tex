%!TeX root = Thesis_LP.tex
\chapter{Conclusion}
Cette thèse propose une approche exhaustive pour la caractérisation sismique de
l'injection du \ce{CO2}, de la modélisation sismique par profilage sismique
vertical (PSV) à l'optimisation stochastique des propriétés élastiques du
réservoir basée sur les simulations d'écoulement. Les principaux développements
méthodologiques ciblant ces différents aspects ont été présentés au sein de
cette partie synthèse. L'approche présentée favorise l'assimilation des données
statiques (mesures de laboratoire, données de forage) avec les données dynamiques
(écoulement du \ce{CO2}).\\

Au niveau de la modélisation sismique de l'injection due \ce{CO2}, des mesures de
laboratoires ont été effectuées afin d'évaluer la réponse sismique ultrasonique du
à l'injection due \ce{CO2} dans deux échantillons du réservoir ciblé pour le
stockage du dioxyde de carbone. \\
Un modèle stochastique calibré sur les résultats obtenus en laboratoire ainsi
que sur les données de forage disponibles dans la zone d'étude a été construit
afin d'effectuer une modélisation (PSV) poroviscoélastique avant et après
injections du \ce{CO2}. Cette approche comporte les étapes suivantes:
\begin{enumerate}[-]
\item  mesurer en laboratoire la valeur du module d'incompressibilité de la roche sèche qui est un
paramètre important dans les équations poroviscoélastiques pour la modélisation
sismique.
\item utiliser un modèle stochastique pour la modélisation sismique, permettant de
reproduire la variation naturelles de la distribution des paramètres à
l’intérieur des différentes formations géologiques et donc d'obtenir une réponse
sismique plus réaliste comparée à la réponse obtenue avec un modèle homogène par
couche;
\item utiliser l'hypothèse d'équilibre vertical, permettant la simulation rapide et
précise de l'injection et de la migration du \ce{CO2}, comparé aux méthodes
numériques classiques qui ne sont pas toujours abordables ou même possibles à
mettre en œuvre car ces dernières requièrent des efforts de calculs élevés;
\item utiliser le profilage vertical pour augmenter la résolution sismique
verticale. Ceci est particulièrement utile quand les différences de signature sismique entre les
acquisitions temporelles sont très faibles;
\item utiliser une formulation poroviscoélastique est probablement l'outil le plus
efficace pour étudier l'effet de la saturation des fluides sur la réponse
sismique car avec cette formulation, les propriétés des fluides sont directement
intégrées dans les équations et les intéractions fluides/matrices qui sont prises en compte par le module de couplage.
\end{enumerate}

Les résultats obtenus ont montré que pour un contexte tel que celui qui a été étudié,
c'est-à-dire avec des porosités et des perméabilités très faibles, les différences
dans la réponse sismique dues à l'injections du \ce{CO2} sont relativement faibles
et se résument en un délai dans le temps d'arrivée de l'onde, de l'ordre de
\SI{30}{\milli\second}. La comparaison avec un scénario optimal (avec des plus grandes porosités et des pérmeabilites) montre que malgré que le panache du \ce{CO2} reste confiné autour du puits d'injection, la réponse sismique est comparable.\\
L'analyse de la variation des amplitudes avec le déport
des tirs a été faite uniquement avec des déports courts car l'apparition d'une onde
réfractée pour les déport supérieurs à \SI{700}{\metre} a empêché la séparation de
ondes ascendantes des ondes descendantes. Il va sans dire qu'il reste des idées
de recherche à poursuivre:
\begin{enumerate}[-]
\item mesurer les effets de la pression dû à l'injection du \ce{CO2} en
laboratoire et les transposer à l'échelle du réservoir afin de pouvoir les séparer
des effet dus à la substitution de fluide;
\item appliquer la méthodologie à des données réelles;
\item adapter cette méthodologie pour des scénarios 3D.
\end{enumerate}
\vspace{10 mm}
Concernant la modélisation stochastique d'un réservoir potentiel pour la
séquestration géologique du \ce{CO2}, une approche en trois étapes à été proposée
pour l'optimisation des modèles de réservoir, basées sur les données statiques
(c'est-à-dire les données de forages) et les données dynamiques (c'est-à-dire la simulation
d'écoulement du \ce{CO2}). Quelques constats peuvent être fait à cet égard:
\begin{enumerate}[-]
\item l'utilisation des données dynamiques pour la caractérisation de réservoir
pour la séquestration du \ce{CO2} apporte un bénéfice majeur pour l'obtention
d'un modèle final optimal;
\item la modélisation de l'onde complète pour plusieurs déports à court, moyen
et longue distances permet de tenir compte de toutes les variations dues à l'injection
du \ce{CO2};
\item l'utilisation d'un processeur graphique (GPU) pour la modélisation sismique à
chaque itération permet de réduire considérablement (\num{2} ordres de
grandeurs) les temps de calculs.
\end{enumerate}

Les résultats obtenus ont montré que l'utilisation des données dynamiques
dans la boucle d'optimisation permet d'améliorer la correspondance sismique du
modèle de réservoir simulé avec le modèle de référence. Cependant des
développements futurs qui permettront d'accroître sa portée sont envisageables:
\begin{enumerate}[-]
\item cette approche a été uniquement testée avec un modèle synthétique réaliste.
Son application à des données réelles permettrait de valider l'approche;
\item adapter cette méthodologie pour des scénario 3D.
\end{enumerate}
