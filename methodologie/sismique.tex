%!TeX root = Thesis_LP.tex
\chapter{La sismique comme outil de surveillance du
\texorpdfstring{\ce{CO2}}{CO2}}
\label{ch:sismique}
Cette section décrit brièvement les bases théoriques et les démarches
méthodologiques nécessaires pour répondre au premier objectif de l’étude. La
\cref{sc:theorie_sismique} présente la base théorique de la propagation des
ondes acoustiques dans les milieux poreux. Les détails de la méthodologie
utilisée se retrouvent dans l’article I faisant partie de cette thèse. Afin
d'éviter la redondance, les \cref{sc:laboratoire,,sc:model} présentent un survol
sur la méthodologie utilisée pour les mesures de laboratoires et pour la
modélisation sismique de l'injection du \ce{CO2} et une discussion sur les
résultats.
% \section{La sismique comme outil de surveillance du
% \texorpdfstring{\ce{CO2}}{CO2}}
% \label{sc:sismique}
\section{Concepts théoriques de base}
\label{sc:theorie_sismique}
\subsection{Les milieux élastiques}
 On appelle onde sismique toute onde mécanique qui traverse un milieu
géologique. Dans l'analyse des données sismiques, on utilise souvent
l'approximation d'élasticité, c'est à dire que les particules reviennent à leur
place après le déplacement imposé par l'onde mécanique.\\
La théorie de l'élasticité part du principe que si un solide est soumis à des
contraintes, il se déforme et lorsque la contrainte est retiré il reprend sa
forme initiale. En sismique, on impose a priori que forces et déformations
sont minimes et donc que les relations entre forces et déformations sont
linéaires, ce qui permet de représenter le milieu comme parfaitement élastique
où toute l'énergie est conservé \citep{Sheriff1995}. Il existe deux types de
contraintes, définis par l’orientation selon lesquelles la force est exercée. Si
la force est appliquée perpendiculairement à la surface, on parle de contrainte
normale, si elle est appliquée de façon tangentielle, on parle de contrainte de
cisaillement. Le comportement mécanique d’un milieu élastique, anisotrope et
linéaire peut être décrit par la loi de Hooke généralisée:
\begin{equation}
\sigma_{ij} = C_{ijkl}*\epsilon_{kl} \qquad i,j,k,l = 1,2,3
\label{eq:hooke}
\end{equation}
où $\sigma$ représente la contrainte, $C$ est le tenseur de rigidité de la
matrice et $\epsilon$ est la déformation.
La contrainte et la déformation peuvent être representée par des matrices $3
\times 3$ (9 composantes) qui représentent la tridimensionnalité d’un volume. Le
comportement du milieu peut donc être modélisé par un tenseur de rigidité de
81 composantes ($3 \times 3 \times 3 \times 3$) qui sont réduites à 21 grâce à
la symétrie entre la contrainte et la déformation. Il s’agit du nombre maximal
de composantes qu’un milieu homogène linéaire peut avoir. Un milieu isotrope,
qui présente la symétrie maximale, est caractérisé par deux composantes
indépendantes, tandis que les milieux avec une symétrie triclinique sont
représentés par l'ensemble des 21 composantes.\par

C’est une pratique courante d’utiliser la notation de Voigt pour représenter les
contraintes, les déformations et les tenseurs de rigidité. Avec cette notation,
les contraintes et les déformations deviennent des vecteurs de six éléments
plutôt que des matrices carrées de 9 éléments. Avec la notation de Voigt, les 4
indices du tenseur de rigidité sont réduits à deux, en utilisant la convention
suivante:\par
$$\begin{matrix}
ij(kl) & I(J)\\
 11 & 1 \\
 22 & 2 \\
 33 & 3 \\
 23, 32 & 4 \\
 13, 31 & 5 \\
 12, 21 & 6 \\
\end{matrix}$$
En utilisant la notation de Voigt, on peut écrire l’\cref{eq:hooke} sous cette
forme:
\[
    \begin{bmatrix}
        \sigma_{1} \\
        \sigma_{2} \\
        \sigma_{3} \\
        \sigma_{4} \\
        \sigma_{5} \\
        \sigma_{6} \\
    \end{bmatrix}
    =
    \begin{bmatrix}
        C_{11} & C_{12} & C_{13} & C_{14} & C_{15} & C_{16} \\
        C_{12} & C_{22} & C_{23} & C_{24} & C_{25} & C_{26} \\
        C_{13} & C_{23} & C_{33} & C_{34} & C_{35} & C_{36} \\
        C_{14} & C_{24} & C_{34} & C_{44} & C_{45} & C_{46} \\
        C_{15} & C_{25} & C_{35} & C_{45} & C_{55} & C_{56} \\
        C_{16} & C_{26} & C_{36} & C_{46} & C_{56} & C_{66} \\
    \end{bmatrix}
    \begin{bmatrix}
        \epsilon_{1} \\
        \epsilon_{2} \\
        \epsilon_{3} \\
        \epsilon_{4} \\
        \epsilon_{5} \\
        \epsilon_{6} \\
    \end{bmatrix}
.\]
Dans le cas isotrope, la matrice de rigidité s’écrit comme suit:
\[
    \begin{bmatrix}
        C_{11} & C_{12} & C_{12} &   0    &   0    &    0   \\
        C_{12} & C_{11} & C_{12} &   0    &   0    &    0   \\
        C_{12} & C_{12} & C_{11} &   0    &   0    &    0   \\
           0   &   0    &   0    & C_{44} &   0    &    0   \\
           0   &   0    &   0    &   0    & C_{44} &    0   \\
           0   &   0    &   0    &   0    &   0    & C_{44} \\
    \end{bmatrix}
, \qquad C_{12} = C_{11} - 2C_{44}
\]
Les relations entre les constantes élastiques $C$  et les paramètres de Lamé
$\lambda$ et $\mu$ pour un milieu isotrope sont:
\begin{equation}
C_{11} = \lambda + 2\mu = K + \frac{4}{3}\lambda, \qquad C_{12} = \lambda,
\qquad C_{44}= \mu,
\end{equation}
où $K$ est le \textbf{module d'incompressibilité} et $\lambda$ le \textbf{module
de cisaillement} du milieu.
En sismique, la propagation des ondes est généralement donnée en termes de module
d'incompressibilité et de cisaillement car ils ont des interprétations physiques
claires. Le premier est essentiellement la mesure de la résistance du milieu à
une compression uniforme (rigidité). Le module de cisaillement, ou deuxième
paramètre de Lamé est une mesure de la résistance du milieu à une déformation en
cisaillement. Le premier paramètre de Lamé, $\lambda$, n'a aucune interprétation
physique, mais il intervient dans la simplification de la matrice de rigidité.\par
D'autres modules peuvent être utilisés pour décrire un milieu isotrope sous une
contrainte uniaxiale. Le \textbf{module de Young} $E$ est le rapport entre la
contrainte appliquée et l'allongement relatif. Le \textbf{coefficient de
Poisson} est le rapport entre la déformation transversale et axiale. Dans le cas
d'une déformation uniaxiale, on peut utiliser le \textbf{module des ondes P}
défini comme le rapport entre la contrainte et la déformation axiale. Pour de
plus amples détails sur le sujet des milieux élastiques voici quelques
références: \citet{Bourbie1986,Carcione2007,Mavko2009}
\subsection*{La propagation des ondes dans les milieux élastiques}
\label{sc:prop_ondes}
Dans la section précédente, la relation entre contrainte appliquée et
déformation a été établie en utilisant la loi de Hooke. Cependant, cette loi ne
donne pas la variation du déplacement des points du milieu avec le temps. La
propagation d’une onde dans l’espace et dans le temps peut être décrite si le
volume du milieu considéré n’est pas en équilibre statique. La deuxième loi du
mouvement de Newton indique qu’une force non nulle exercée sur un corps est
égale au produit de la masse et de l’accélération du corps. En incluant la loi
de Hooke dans l’équation du mouvement et en exprimant la déformation en terme de
déplacement, l’équation d’onde à une dimension dans un milieu élastique pour un
déplacement $u$ est donnée par:
\begin{equation}
\rho\frac{\partial^2 u}{\partial t^2} = C\nabla^2u,
\label{eq:onde}
\end{equation}
où $u$ est fonction de la position et du temps, $\rho$ est la
densité\footnote{À strictement parler il s'agit de la masse volumique, mais en
géophysique on utilise couramment le terme de densité.} du milieu
élastique et $C$ et la constante de rigidité ou module élastique relatif au
type d'onde pris en considération. La vitesse de l'onde pour le cas le plus
général de l'\cref{eq:onde} est:
\begin{equation}
V = \sqrt{\frac{C}{\rho}}
\label{eq:velocity_gen}
\end{equation}
Essentiellement, l’équation d’onde met en relation la dérivée dans le temps avec
la dérivée dans l’espace du déplacement par la constante de proportionnalité de
$V^2$.\\
Dans un milieu homogène isotrope, il y a 2 types d’ondes principales qui sont
étudiées: les ondes de compression $P$ et les ondes de cisaillement $S$. Pour la
vitesse des ondes $P$ et $S$, l'\cref{eq:velocity_gen} devient:
\begin{equation}
V_p = \sqrt{\frac{C_{11}}{\rho}} = \sqrt{\frac{\lambda + 2\mu}{\rho}} =
\sqrt{\frac{K+\frac{4}{3}\mu}{\rho}}\quad et
\label{eq:vitesse_p}
\end{equation}
\begin{equation}
V_s = \sqrt{\frac{C_{44}}{\rho}} = \sqrt{\frac{\mu}{\rho}} ,
\label{eq:vitesse_s}
\end{equation}
respectivement. Les fluides ne peuvent pas soutenir les forces de cisaillement,
le module de cisaillement des fluides est donc égal à zéro. Seules les ondes
$P$ peuvent voyager dans des liquides dont la vitesse de l’onde est:
\begin{equation}
V_p = \sqrt{\frac{K}{\rho}} .
\label{eq:velocity_pliq}
\end{equation}
La théorie présentée jusqu'ici s'applique à des milieux monophasiques (solides).
Pour de plus amples détails sur la propagation des ondes dans ce type de milieu,
voici quelques références: \citet{Sheriff1995,Aki1980}.\par

Pour étendre l'étude de la propagation des ondes aux milieux poreux saturés
des fluides, on peut appliquer le modèle de Biot \citep{Biot1956a,Biot1956b}. Dans ce
modèle, les interactions fluide-structure sont prises en compte à travers trois
types de couplage : les couplages massiques, élastiques et visqueux.
\subsection{Les milieux viscoélastiques}
Le comportement viscoélastique est une réponse mécanique, en fonction du temps,
d'un milieu à des variations des contraintes appliquées. \citet{Boltzmann1874} a
été parmi les premiers scientifiques à introduire le concept de mémoire: pour un
point donné d'un milieu, la contrainte appliquée au temps $t$ dépend de la
déformation du milieu au temps $t-1$. Différemment d'un milieu purement
élastique, où l'énergie utilisée pour déformer le milieu est conservée, dans un
milieu viscoélastique, elle est partiellement dissipée. N'ayant plus d'énergie
pour retourner jusqu'à l'état initial, un milieu viscoélastique reste donc déformé
\citep{Carcione2007}. \\
Le facteur adimensionnel de qualité $Q$ permet de caractériser la dissipation
d'un milieu. Par définition, le facteur Q est inversement proportionnel à
l’énergie absorbée par le milieu lors d’un cycle d’oscillation de l’onde
\citep{Sheriff1995}.  Cette définition revêt la forme
mathématique suivante:
\begin{equation}
Q = \dfrac{2\pi}{\Delta E /E}.
\label{eq:facteur_Q}
\end{equation}
Ainsi, plus le matériau est de piètre qualité du point de vue sismique, plus
l’énergie de l’onde sismique dissipée ($\Delta E$) est grande, plus le facteur
de qualité sera faible \citep{Giroux2001}. Pour un système viscoélastique, il y
a une relation directe entre la vitesse de dispersion et le facteur de qualité.
\subsection{Les milieux poroviscoélastiques}
Le concept de viscoélasticité peut être introduit dans les équations de Biot
\citep{Biot1956a,Biot1956b} pour la modélisation des mécanismes d'atténuation
liée à l'énergie de déformation (dissipation due à la rigidité) et à l'énergie
cinétique (dissipation viscodynamique) \citep{Carcione2007}.
\citet{Carcione1998} a modifié les équations de Biot afin d'inclure des
mécanismes d'interaction matrice-fluide à travers les fonctions de relaxation
viscoélastique. Cette formulation est la plus appropriée pour décrire le
phénomène de l'injection du \ce{CO2} dans des grès, car il permet d’inclure les
propriétés des fluides dans les équations de propagation des ondes.\\
Les développements mathématiques pour la propagation des ondes dans les milieux
viscoélastiques et poroviscoélastiques sont détaillés dans \citet{Bourbie1986}
et \citet{Carcione2007}
\subsection{La physique des roches}
La discipline de la physique des roches a comme objectif d'établir des relations
entre les propriétés physiques des roches et leur réponse géophysique mesurée. Dans notre cas,
la physique des roches étudie les propriétés physiques qui influencent la
propagation des ondes sismiques à travers les roches, à savoir la
compressibilité, la rigidité, la porosité, la densité et les fluides
interstitiels. Pour établir ces relations, il faut connaître les propriétés
élastiques de la matrice et des fluides interstitiels ainsi que les modèles
d'interaction entre fluide et roche. \\
Le concept de substitution de fluide se réfère à la modélisation des vitesses
des ondes sismiques dans un milieu poreux saturé, en faisant varier les propriétés physiques des fluides intersticiels.
\subsubsection{L'équation de Gassmann}
En physique des roches, la relation de Gassmann \citep{Gassmann} est fréquemment
utilisée en raison de sa simplicité et de son applicabilité dans la gamme des fréquences
sismiques, autour de \SI{100}{\hertz} \citep{Mavko2009}. Dans sa formulation,
Gassmann, fait plusieurs hypothèses:
\begin{enumerate}
\item La roche est considérée comme homogène et isotrope;
\item Les minéraux constituant la roche ont le même module de rigidité et de
cisaillement;
\item Les fluides interstitiels peuvent circuler librement et les pores sont
connectés entre eux;
\item Les pores sont complètement saturés;
\item Les fluides interstitiels n'interagissent pas avec les minéraux formant la matrice rocheuse;
\item Les fréquences sont suffisamment basses pour que la pression induite dans
les pores puisse se rééquilibrer.
\end{enumerate}
Dans l’équation de Gassmann, le module de la roche saturée $K$ est relié au
module de la matrice (roche sèche) $K_{dry}$, le module de la roche solide
(minéraux constituant la roche) $K_s$, le module du fluide interstitiel $K_f$ et
la porosité de la roche $\phi$ par \citep{Mavko2009}:
\begin{equation}
\dfrac{K}{K_s - K} = \dfrac{K_{dry}}{K_s - K_{dry}} + \dfrac{K_f}{\phi (K_s -
K_f)},
\label{eq:gass}
\end{equation}
en réarrangeant l'\cref{eq:gass}, on a:
\begin{equation}
K = K_{dry} + \dfrac{\bigg(1 -
\dfrac{K_{dry}}{K_s}\bigg)^2}{\dfrac{\phi}{K_f}+\dfrac{1-\phi}{K_s} +
\dfrac{K_{dry}}{K_s^2}}.
\label{eq:gasmmann}
\end{equation}
Si le module de la roche sèche $K_{dry}$ n'est pas disponible, le module de la
roche saturée  $K$ peut être lié au module de la roche saturée avec un autre
fluide $K_2$ selon la relation suivante\citep{Mavko2009}:
\begin{equation}
\dfrac{K}{K_s - K} - \dfrac{K_{f}}{\phi(K_s - K_{f})} = \dfrac{K_2}{K_s - K_2} -
\dfrac{K_{f2}}{\phi(K_s - K_{f2})}.
\end{equation}
Dans la formulation de Gassmann, le module de cisaillement est indépendant des
fluides qui saturent la roche, car ces derniers sont incapables de soutenir des
forces de cisaillement. Donc, le module de cisaillement de la roche saturée est
égal au module de cisaillement de la roche sèche:
\begin{equation}
\mu_{sat} = \mu_{dry}
\end{equation}
À partir des résultats de $K$ et $\mu$, les vitesses $V_p$ et $V_s$
correspondantes peuvent être calculées en utilisant les
\cref{eq:vitesse_p,,eq:vitesse_s} où la densité de la roche saturée est:
\begin{equation}
\rho_{sat} = (1-\phi)\rho_s + \phi \rho_f.
\end{equation}
Avant d'effectuer la substitution de fluide en utilisant l'\cref{eq:gasmmann},
il faut déterminer la porosité ($\phi$), les propriétés des fluides
interstitiels ($K_f$, $\rho_f$), le module de la matrice ($K_{dry}$) ainsi que le
module des solides ($K_s$) de la roche. Ces quatre composantes peuvent être
inférées à partir des mesures de laboratoire ou par l'analyse de diagraphies en forage. Une
revue exhaustive des méthodes permettant de déterminer $\phi$, $\rho_f$, $K_f$, $K_{dry}$, $K_s$ se trouve dans
\citet{Smith2003} et \citet{Mavko2009}.
\subsubsection{La formulation de Biot}
Contrairement à Gassmann, Biot prédit la dépendance en fréquence des vitesses
des ondes dans les milieux saturés. Il a présenté sa théorie pour les basses et
les hautes fréquences dans \citet{Biot1956a,Biot1956b}. Pour les basses
fréquences, la relation de Biot se réduit à celle de Gassmann. Les mêmes
hypothèses que pour Gassmann s'appliquent. De plus, Biot suppose que les fluides
interstitiels sont newtoniens (c'est-à-dire que loi contrainte – vitesse de déformation est linéaire, et où la constante de proportionnalité est la viscosité). Dans sa formulation, Biot représente la
dépendance en fréquence des ondes en incorporant les interactions visqueuses et
inertielles entre le fluide interstitiel et la matrice solide de la roche. \\
Pour les hautes fréquences, les fluides interstitiels n'ont pas assez de temps
pour se rééquilibrer donnant naissance à des phénomènes d’atténuation et
dispersions connus sous le nom d'écoulement de fluide induit par la propagation
des ondes, de l'anglais \emph{wave-induced fluid flow} \citep{Muller2010}.\\
Les bases mathématiques de la formulation de Biot sont développées dans les
articles originaux \citet{Biot1956a,Biot1956b} ainsi que dans
\citet{Bourbie1986}, \citet{Carcione2007} et \citet{Allard2009}.
\section{Mesures sismiques de laboratoire avec injection de
\texorpdfstring{\ce{CO2}}{CO2}}
\label{sc:laboratoire}
La première étape de ma thèse a été de vérifier et de mesurer la relation entre
les propriétés physiques et géologiques dans les conditions de pression et
température des réservoirs potentiels du Québec. Les mesures ont été
 effectué un stage de 3 mois dans le laboratoire du professeur Schmitt à
l'Université d'Edmonton.
La méthode de la transmission par impulsion est parmi les méthodes ultrasoniques
les plus utilisées en physique des roches \citep{Wyllie1958,Nur1971,Timur1977,Toksz1979,Tosaya1982,Blair1990,Wang1991,Cadoret1995,Adam2006,Verwer2008,Yam2011,Njiekak2013,Schmitt2015}
et elle est la seule méthode appliquée à ce jour pour les études en laboratoire
du \ce{CO2} sur les ondes élastiques. En comparaison avec d'autres techniques,
la transmission par impulsion est relativement simple et facilement applicable.
Des variables telles que la pression, la température et la saturation peuvent
être manipulées pour étudier leur effet sur la réponse sismique. \\
L'approche proposée dans cette étude s'inspire directement de
\citet{Schmitt2015} et les réponses sismiques associées aux différentes phases
du \ce{CO2} ont été étudiées sur deux échantillons, un du Covey Hill et un du Cairnside,
complètement saturés en \ce{CO2} avec la méthode de transmission par impulsion.
Avec cette méthode, l'échantillon est placé entre la source et un récepteur qui
sont généralement des transducteurs piézoélectriques en céramique. La
\cref{sc:art_1_ultrasonic_measurements} de l'article I à la
\cpageref{sc:art_1_ultrasonic_measurements} décrit la procédure des mesures de
laboratoire, de la préparation des échantillons jusqu'aux analyses. Les
paragraphes qui suivent donnent un aperçu des étapes principales.
\subsection{Préparation des échantillons}
Deux échantillons cylindriques, de \SI{3.7}{\cm} et de \SI{4}{\cm} de longueur,
ont été préparés pour les analyses. Un aspect très important pour améliorer la
transmission du signal et pour minimiser les erreurs de mesure est de s'assurer
que les extrémités de l'échantillon soient les plus parallèles possible entre
elles. Les échantillons ont été initialement taillés afin de les rendre
approximativement parallèles et ensuite ont été polis en utilisant une meuleuse
afin que le parallélisme entre les extrémités soit de l'ordre de
\SI{\pm0.025}{\mm}. Avant de commencer les mesures, les deux échantillons ont
été séchés dans une étuve à \SI{70}{\degreeCelsius} pendant \numrange{24}{36}
heures et déposés dans un dessiccateur.\\
L'étape finale de la préparation consiste à sceller l'assemblage avec un tube
Tygon\texttrademark{} qui assure l’étanchéité à l'huile hydraulique présente à
l’intérieur de la cuve sous pression où les mesures sont effectuées. La
\cref{fig:apparatus} de l’article I à la \cpageref{fig:apparatus} montre
l'assemblage de l'échantillon avec les transducteurs scellés dans le tube de
Tygon\texttrademark{} prêt pour être introduit dans la cuve sous pression
(désigné avec la lettre A dans la même figure). Cette figure montre aussi le
système de pompage pour régler la pression de confinement et interstitielle. La
\cref{sc:experimental_apparatus} de l'article I à la
\cpageref{sc:experimental_apparatus} décrit les caractéristiques des différentes
parties de l'équipement de mesure utilisé dans le cadre de ma thèse.
\subsection{Procédure expérimentale}
Les échantillons ont été soumis à une série de mesures, y compris des mesures en
conditions sèches et différentes conditions de saturation en \ce{CO2}. Avant de
décrire les mesures, il est important de définir les types de pression qui
peuvent être appliqués aux échantillons. Comme décrit dans la section
précédente, le système des pompes peut contrôler deux types de pressions; la
pression appliquée à la superficie de l’échantillon (pression de confinement) et
la pression du fluide interstitiel (pression de pore). Ces deux types de
pressions sont exercées dans des directions opposées et l’on définit la pression
différentielle $P_d$ comme suit:
\begin{equation}
P_d = P_c - P_p,
\end{equation}
où $P_c$ est la pression de confinement qui est généralement plus élevée que la
pression de pore $P_p$. Dans le cas où $P_p$ est plus grande que $P_c$ on a
fracturation hydraulique de la roche.\\
Les caractéristiques des mesures effectuées pour les échantillons du Covey Hill
et du Cairnside sont résumées dans le \cref{tbl:measures}.
\begin{table}[tb]
    \caption{Type de mesures effectuées sur les échantillons du Covey Hill et du
Cairnside.}
    \label{tbl:measures}
    \sisetup{per-mode = symbol,table-format = 1.2e-2}
    \centering
        \begin{tabular}{ccccccc}
            \toprule
            \multirow{2}{*}{Type de mesure} & {Température} &
\multicolumn{5}{c}{Pression (\si{\mega\pascal})}\\
            \cmidrule(r){3-7}
             & {(\si{\degreeCelsius})}   & {\footnotesize{Confinement}} & -
&{\footnotesize{Pore}} & = & {\footnotesize{Différentielle}} \\
            \midrule
            \rowcolor{Gray}& 23 & \numrange{3}{45} & & 0 & &\numrange{3}{45} \\
            \rowcolor{Gray}\multirow{-2}{*}{Sèche}  & \numrange{23}{45} & 14 &&
0 && 14 \\
            & 25 & \numrange{16}{39}& & \numrange{2}{25}& & 14 \\
            & 35 & \numrange{16}{39}& & \numrange{2}{25}& & 14 \\
            \multirow{-3}{*}{\ce{CO2}} & \numrange{27}{50} & 28& & 14& & 14 \\
            \bottomrule
        \end{tabular}
\end{table}\\
La première série de mesures a impliqué les échantillons secs. Ces mesures ont
été effectuées avec un cycle sous pression suivi d'un cycle de dépressurisation
pour vérifier des éventuels changements dans la structure de la roche sèche
lorsqu'elle est soumise à des contraintes de pression élevées. De plus, ces
mesures permettent d'obtenir le module de la roche sèche $K_{dry}$ en utilisant
les \cref{eq:vitesse_p,,eq:vitesse_s}.\\
À la suite des mesures sèches, des mesures avec saturation en \ce{CO2} ont été
effectuées sous différentes contraintes de pression et température. Pour chaque
échantillon, deux températures constantes (\SIlist{25;35}{\degreeCelsius}) ont
été utilisées tandis que la pression de pore variait de
\SIrange{2}{25}{\mega\pascal}.
En utilisant ces contraintes, le \ce{CO2} peut se retrouver dans la phase
gazeuse, liquide ou supercritique. Les \cref{fig:density,,fig:bulk} de l'article
I à la \cpageref{fig:bulkdensity} montrent les diagrammes de phase de la densité
et du module du \ce{CO2} ainsi que les conditions de température et pression
auxquels les mesures ont été effectuées.
\subsection{Analyse de la vitesse et de l’amplitude du signal}
Pour chaque mesure, de nombreuses acquisitions ("stacks") des ondes sismiques
$P$ et $S$ ont été enregistrées. À partir de ces acquisitions, les vitesses et
l’amplitude du signal des ondes $P$ et $S$ peuvent être analysées en fonction des
conditions de mesures.\\
Le temps d'arrivée enregistré pour les ondes $P$ et $S$ est une combinaison du
temps nécessaire au signal pour traverser à la fois l’échantillon et les
bouchons d’aluminium. Pour déterminer la vitesse uniquement à travers
l’échantillon, le temps de parcours dans les bouchons d’aluminium doit être
éliminé. Ce temps de parcours est affecté par la pression; des mesures de
calibration ont été donc effectuées sur les bouchons d’aluminiums, pour la gamme
de pressions rencontrées lors des mesures sur les échantillons
(\SIrange{3}{45}{\mega\pascal}).\\
En déterminant la différence de temps d’arrivée du signal à travers les bouchons
d’aluminium avec l’échantillon ($t_{be}$) et le signal à travers uniquement les
bouchons d’aluminium ($t_b$), le temps de parcours du signal à travers
l’échantillon ($t_e$) peut être facilement déterminé. Finalement, la vitesse du
signal $v$ à travers l’échantillon est simplement calculée à partir de $t_e$ et
de la longueur de l’échantillon $l_e$ selon la relation:
\begin{equation}
v = \frac{l_e}{t_e} = \frac{l_e}{(t_{be}-t_b)}.
\end{equation}
Pour chaque mesure, l'amplitude relative du signal a été enregistrée. La
\cref{fig:waveform_b} de l'article I à la \cpageref{fig:waveform_b} montre la
technique utilisée pour analyser l'amplitude du signal. L'amplitude maximale a
été calculée entre le premier pic négatif et le premier pic positif du signal, à
partir de la première arrivée.
\subsection{Synthèse des résultats}
La \cref{fig:results_lab} de l'article I à la \cpageref{fig:results_lab} montre
les vitesses et les amplitudes des ondes $P$ et $S$ pour les mesures
ultrasoniques à \SIlist{25;35}{\degreeCelsius} effectuées sur les échantillons
du Cairnside et du Covey Hill saturés en \ce{CO2}.
\begin{enumerate}[-]
\item Les vitesses des ondes $P$ (\cref{fig:results_lab_a}) diminuent avec
l'augmentation de la pression des pores dans l'intervalle
\SIrange{2}{7}{\mega\pascal}.
\item Une fois que la transition de phase du \ce{CO2} est réalisée (transition
gazeuse à liquide/supercritique), les vitesses augmentent avec la pression des
pores. Cette tendance est beaucoup plus prononcée pour l’échantillon du Covey
Hill que pour celui du Cairnside.
\item Les faibles changements de vitesse enregistrés sur les ondes $S$
(\cref{fig:results_lab_c}) confirment que la différence de vitesse observée sur
les ondes $P$ est due à la transition gazeuse à liquide/supercritique du
\ce{CO2}.
\item L'amplitude du signal pour les ondes $P$ et $S$
(\cref{fig:results_lab_b,,fig:results_lab_d}) montre une diminution rapide dans
l'intervalle \SIrange{5}{7}{\mega\pascal}. Cette tendance est valide uniquement
pour l'échantillon du Covey Hill, tandis que l’échantillon du Cairnside ne
montre presqu'aucune variation d'amplitude.
\end{enumerate}
Les \cref{fig:results_lab_a,,fig:results_lab_c} montrent aussi les vitesses
modélisées en utilisant la relation de Gassmann. Il y a un accord général entre
les vitesses modélisées et mesurées, cependant Gassmann prédit de plus hautes
vitesses lorsque le \ce{CO2} est gazeux et des vitesses plus faibles lorsque le
\ce{CO2} est liquide ou supercritique. Le modèle de Gassmann suppose que le
réseau de pores est connecté. La \cref{fig:poresize} montre que les
échantillons du Cairnside et du Covey Hill ont une faible taille de pores qui
pourrait empêcher la formation d’un tel réseau et donc limiter l’applicabilité
de ce modèle. De plus, Gassmann suppose que la pression au niveau des pores est
en équilibre. C’est possible que pendant les mesures, la pression des pores et
la température n’aient pas eu assez de temps pour se stabiliser et donc affecter
les vitesses.
\section{Modélisation sismique de l'injection du \texorpdfstring{\ce{CO2}}{CO2}}
\label{sc:model}
La deuxième étape importante de cette thèse était de vérifier, sur un modèle
numérique, le potentiel du profilage sismique vertical (PSV) comme outil de
surveillance de la propagation du \ce{CO2} dans des conditions de très faible
porosité. Les paragraphes qui suivent présentent un aperçu des résultats obtenus
avec le profilage sismique vertical ainsi que sur l'utilisation d'un algorithme
pour la propagation des ondes sismiques dans les milieux poroviscoélastiques.
\subsection{Profilage sismique vertical}
Le profilage sismique vertical (Vertical Seismic Profiling) est une méthode
sismique bien adaptée aux petits projets de captage et stockage du carbone car elle
permet de fournir une information de haute résolution \citep{Yang2014} et elle
est économiquement plus avantageuse que le suivit sismique 3D de surface utilisé
couramment dans le domaine pétrolier. En effet, le fait d'avoir des capteurs dans
un puits dans le réservoir permet d'augmenter grandement la résolution. En
revanche, c'est une méthode qui nécessite un puits, mais, une fois que celui-ci
est creusé, il est très rapide et aisé de faire des mesures dans le temps. Le
PSV a été utilisé pour la surveillance du \ce{CO2}  dans plusieurs projets
pilotes tels que Ketzin \citep{Yang2010}, SACROC \citep{Yang2014,Cheng2010},
Frio \citep{Daley2008}, et Otway \citep{Urosevic2008}.\\
L'acquisition de données PSV implique une source à la surface qui peut être
proche du puits où les géophones sont placés (PSV déport nul) ou à une distance
croissante du puits (PSV avec déport).
\begin{figure}[ht]
\centering
\includegraphics[width=0.8\textwidth]{fig/vsp_3D.pdf}
\caption{Schéma de la géométrie d'acquisition PSV}
\label{fig:vsp_3D}
\end{figure}
La \cref{fig:vsp_3D} montre le schéma d'une acquisition PSV. Une revue
exhaustive de cette méthode est présentée dans \citep{Hardage1992,Mari2003}
\subsection{Modèle géologique}
Le modèle géologique utilisé pour la modélisation sismique a été généré à partir
de données acquises dans plusieurs forages disponibles dans la zone d'étude
\citep{Claprood2012,TranNgoc2014}. À partir des diagraphies, un forage
synthétique représentatif pour la région d'étude a été construit pour les valeurs de $V_p$,
$V_s$, densité et porosité (voir \cref{fig:well-log} de l'article II à la
\cpageref{fig:well-log}). Pour chaque formation, les distributions de $V_p$,
$V_s$, densité et porosité ont été calculées, ainsi que leurs variogrammes
verticaux. À partir de ces distributions, un modèle de référence a été généré en
utilisant une approche par cosimulation séquentielle gaussienne,
méthode couramment utilisée pour la modélisation des propriétés géologiques à
partir de données de forages \citep{Deutsch1998,Doyen2007}. Le modèle pour les vitesses
des ondes $P$ est présenté à la \cref{fig:mod_ref_vp}.
\begin{figure}[ht]
\centering
\includegraphics[width=0.8\textwidth]{fig/mod_ref_vp.pdf}
\caption{Modèle de référence pour $V_p$}
\label{fig:mod_ref_vp}
\end{figure}
Le modèle, qui fait \SI{2000 x 1500}{\metre} avec un pas de \SI{1 x
1}{\metre} pour un total de \num{3} millions de nœuds, est présenté à la
\cref{fig:mod_ref_vp}. Le \cref{tbl:modele} résume les caractéristiques
physiques du modèle.
\subsection{Modélisation sismique}
\label{sc:poroviscoelastique}
Un code poroviscoélastique basé sur les travaux de
\citet{Carcione1995,Carcione1996,Carcione1999} et implémenté par
\citet{Giroux2012} a été utilisé pour générer des sismogrammes synthétiques.
L'objectif était d'étudier la performance du PSV pour détecter les changements
sur le signal sismique dus à l'injection du \ce{CO2}. Cette formulation prend en
compte 12 paramètres, à savoir le module de la roche sèche ($K_{dry}$), le
module des minéraux de la roche ($K_s$), le module des fluides interstitiels
($K_{f}$), la porosité ($\phi$), le module de cisaillement de la roche ($G_s$),
la densité des minéraux de la roche ($\rho_s$), la densité des fluides
interstitiels ($\rho_f$), la tortuosité ($\tau$), la viscosité des fluides
($\eta$), la perméabilité ($\kappa$) et le facteur de qualité ($Q$). Le
\cref{tbl:modelpar} de l'article I à la \cpageref{tbl:modelpar}, résume les 12
paramètres du modèle géologique utilisé pour la modélisation sismique. \\
La \cref{fig:mod_ref_vp} montre la géométrie d'acquisition choisie pour la
modélisation. Les sources sont placées à la surface du modèle avec un déport
allant de \SIrange{100}{2000}{\metre} avec un espacement de \SI{100}{\metre}
pour le premier \num{7} déports.
Les géophones sont déployés sur une profondeur allant de \SIrange{200}{1400}{\metre} avec un espacement de
\SI{5}{\metre}. Sur la \cref{fig:mod_ref_vp}, les géophones qui se trouvent dans
le réservoir (Formation du Covey Hill et Cairnside) sont mis en évidence en
rouge.
L'objectif de la modélisation sismique est de simuler des acquisitions PSV
effectuées \numlist{5;15;50} ans après
injection de \ce{CO2}.
\begin{table}
  \centering
  \caption{Caractéristiques du modèle géologique et géométrie d’acquisition
PSV.}
 \begin{tabular}{p{5cm}c}
\toprule
 {Paramètre}  & {Valeur}   \\
\midrule
% Avg. porosity &(\si{\percent})  &  9.75  &  20   \\
% Avg. permeability & (\si{\metre\squared})  & 3.06e-16  & 6.5e-13  \\
Portée en x  & \SI{2000}{\metre}   \\
Portée en z  & \SI{1500}{\metre}   \\
Nœuds  & \num{3} millions   \\
% Injection depth &(\si{\metre}) & \multicolumn{2}{c}{1200-1350 }  \\
% Injection rate &(\si{\tonne\per\metre}) & \multicolumn{2}{c}{45 }  \\
% Injection time &(years) & \multicolumn{2}{c}{15 }  \\
% Migration time &(years) & \multicolumn{2}{c}{35}  \\
% Total storage &(\si{\kilo\tonne}) & \multicolumn{2}{c}{245}\\
Déport des sources  & \SIrange{100}{700}{\metre}    \\
Espacement des sources  & 1\SI{100}{\metre} \\
Coordonnée x des géophones  & \SI{200}{\metre} \\
Coordonnée z des géophones  & \SIrange{200}{1400}{\metre} \\
Espacement des géophones  & \SI{5}{\metre} \\
Traces totales  & 241 \\
\bottomrule
\end{tabular}
\label{tbl:modele}
\end{table}
\subsubsection{Séquence de traitement}
La séquence de traitement s'inspire du travail de
\citet{Coulombe1996,Zhang2010}. Le \cref{tbl:process} de l'article I à la
\cpageref{tbl:process} et la \cref{fig:traitement} résument les étapes du
traitement. Il s'agit d'un traitement classique pour les données de PSV, il ne
sera pas donc détaillé ici. La séquence de traitement a été appliquée uniquement
au déport inférieur à \SI{700}{\metre}; en effet pour les plus grands déports
(\SIlist{1300;1800} ), une onde réfractée apparaît à l'interface entre la
Formation du Lorraine-Utica et le Groupe Trenton, comme le montre la
\cref{fig:refracted} de l'article I à la \cpageref{fig:refracted} et qui empêche
la séparation des ondes montantes et descendantes et donc le traitement du PSV.
L'analyse des variations des amplitudes avec déports (\emph{Amplitude variation
with offset - AVO}) a été donc limitée aux déports inférieurs à
\SI{700}{\metre}.
\begin{figure}[!ht]
\centering
\includegraphics[width=1\textwidth]{fig/traitement.pdf}
\caption{Illustration de la séquence du traitement pour le Profilage Sismique
Verticale (PSV).}
\label{fig:traitement}
\end{figure}
\subsection{Modélisation de l'écoulement du \texorpdfstring{\ce{CO2}}{CO2}}
\label{sc:ecouelement}
Les approches classiques pour la modélisation de l'injection du \ce{CO2}
utilisent
des méthodes numériques en trois dimensions afin de reproduire avec un degré
élevé de précision les effets de l'hétérogénéité et de la dispersion
\citep{White1997,Pruess1999,Pruess2004,Schlumberger2007}. Cependant, ces
méthodes requièrent des efforts de calculs notables.\\
En partant d'hypothèses simplificatrices, des modèles qui demandent beaucoup moins
d'effort de calcul peuvent être développés. Une de ces hypothèses est d'utiliser
des méthodes semi-analytiques. Ce sont ces méthodes qui ont été les plus développées ces
dernières années
\citep{Nordbotten2004,Nordbotten2005a,Nordbotten2005,Nordbotten2009}. Cette
méthode suppose que l’aquifère est homogène et horizontal, qu'il y a une
interface définie entre les deux fluides (\ce{CO2} injectée et saumure) et que
la géométrie d'injection du \ce{CO2} est à symétrie radiale \citep{Gasda2009}.
Avec ces contraintes, les méthodes analytiques deviennent un outil puissant pour
la modélisation de l'injection et de la migration du \ce{CO2}.\\
Une méthode prometteuse pour la modélisation rapide et précise de la
séquestration du \ce{CO2} est basée sur l'hypothèse de l'équilibre vertical
(VE). Les modèles formulés par VE ont une longue tradition pour la simulation
des processus d'écoulement dans les milieux poreux; en hydrogéologie ils sont
connus comme approximation de Dupuit, tandis que dans l'industrie pétrolière ils
sont utilisés pour la simulation de l’écoulement multiphase
\citep{Martin1958,Coats1967,Martin1968}.\\
Pour modéliser l'écoulement après \numlist{5;15;50} ans du début de l'injection,
le modèle à équilibre vertical développé par \citet{Ligaarden2010} et contenu
dans MRST - \emph{Matlab Reservoir Simulation Toolbox} \citep{Lie2012} a été
utilisé. Dans cette formulation, la saturation moyenne est $s=\frac{h}{H}$ et
correspond à la hauteur relative du panache de \ce{CO2}, comme montré sur la
\cref{fig:VE}.
\begin{figure}[!ht]
\centering
\includegraphics[width=0.7\textwidth]{fig/VE.png}
\caption{Illustration du panache de \ce{CO2} assumé dans les modèles à équilibre
vertical (VE), d'après \citet{Ligaarden2010}.}
\label{fig:VE}
\end{figure}
La \cref{fig:res} montre les distributions de porosité et de perméabilité au
niveau du réservoir (Formation du Cairnside et du Covey Hill), qui sont utilisées
pour simuler l'injection du \ce{CO2}. La pérméabilité est dérivée en utilisant
une extension de l'équation de Kozeny-Carman \citep{Kozeny1927,Carman1938}:
\begin{equation}
k = \dfrac{1}{72}\dfrac{\phi^3}{(1-\phi)^2\tau^2}d^2,
\end{equation}
où $d$ est le diamètre des grains qui composent la roche et $\tau$ la tortuosité. Pour des grès très
compacts, $k$ est de l'ordre \SI{5e-6}{\metre}
\citep{DonaldRWiesnet1961}.
\begin{figure}[!ht]
        \centering
        \begin{subfigure}[b]{1\textwidth}
                \caption{Porosité}
                \includegraphics[width=\textwidth]{fig/phi_res.pdf}
                \label{fig:phi_res}
        \end{subfigure}%

        \begin{subfigure}[b]{1\textwidth}
                \caption{Perméabilité}
                \includegraphics[width=\textwidth]{fig/K_res.pdf}
                \label{fig:K_res}
        \end{subfigure}

        \caption{Distribution de la porosité et de la perméabilité dans le
réservoir.}
        \label{fig:res}.
\end{figure}
La simulation de l'écoulement se fait dans la formation du Covey Hill
(\SIrange{1200}{1350}{\metre}). Les \cref{fig:5r,fig:15r,,fig:50r} montrent
l'extension et la saturation du panache de \ce{CO2} après \numlist{5;15;50}
ans.\par

Un deuxième scénario, avec une porosité moyenne de \SI{20}{\percent} dans le
réservoir (scénario optimal) a été étudié. Les résultats des simulations
d'écoulement de \ce{CO2} sont présentés aux \cref{fig:5o,fig:15o,,fig:50o}. Pour
ce scénario, l'extension du panache est logiquement plus grande. Les paramètres
principaux pour les deux scénarios sont résumés dans le \cref{tbl:co2par} de
l'article I à la \cpageref{tbl:co2par}.
% \begin{landscape}
\begin{figure}[p]
        \centering
        \begin{subfigure}[b]{.47\textwidth}
                \caption{5 ans après injection \\ Scénario réaliste }
                \includegraphics[width=\textwidth]{fig/5r.pdf}
                \label{fig:5r}
        \end{subfigure}
        \qquad
        \begin{subfigure}[b]{.47\textwidth}
                \caption{5 ans après injection \\ Scénario optimal}
                \includegraphics[width=\textwidth]{fig/5o.pdf}
                \label{fig:5o}
        \end{subfigure}

        \begin{subfigure}[b]{.47\textwidth}
                \caption{15 ans après injection }
                \includegraphics[width=\textwidth]{fig/15r.pdf}
                \label{fig:15r}
        \end{subfigure}
        \qquad
        \begin{subfigure}[b]{.47\textwidth}
                \caption{15 ans après injection  }
                \includegraphics[width=\textwidth]{fig/15o.pdf}
                \label{fig:15o}
        \end{subfigure}

        \begin{subfigure}[b]{.47\textwidth}
                \caption{50 ans après injection }
                \includegraphics[width=\textwidth]{fig/50r.pdf}
                \label{fig:50r}
        \end{subfigure}
        \qquad
         \begin{subfigure}[b]{.47\textwidth}
                \caption{50 ans après injection}
                \includegraphics[width=\textwidth]{fig/50o.pdf}
                \label{fig:50o}
        \end{subfigure}

        \caption{Saturation du panache du \ce{CO2} en fonction du temps.}
        \label{fig:co2_sat}.
\end{figure}
% \end{landscape}
\subsection{Synthèse des résultats}
Cette section présente une synthèse des résultats obtenus, qui sont discutés plus
en détail dans l'article I.
Les \cref{fig:seism_opt_a,fig:seism_opt_b,fig:seism_opt_c,fig:seism_bec_a,fig:seism_bec_b,fig:seism_bec_c}
de l'article I aux \cpageref{fig:seismopt,fig:seismbec} montrent les résultats
pour les suivis à \numlist{5;15;50} ans des scenarii, respectivement optimal et réaliste. Les deux figures montrent la sommation des traces pour les
déports allant de \SIrange{100}{700}{\metre}.
Pour chaque suivi, la comparaison avec le suivi précédent ainsi que leur
différence sont montrés.\\
Pour les deux scénarios, un délai dû à l'injection du \ce{CO2} est détecté à
partir de \SI{550}{\milli\second}. Ce delai est de l'ordre de
\SI{30}{\milli\second} à la base du réservoir
(\cref{fig:seism_opt_a,fig:seism_bec_a}), comme décrit dans le \cref{tbl:delay}
de l'article I à la \cpageref{tbl:delay}. Le suivi à \num{15} ans pour le
scénario optimal ne montre aucune différence avec le suivi à \num{5} ans
\cref{fig:seism_opt_b}. Pour la période de migration du \ce{CO2}
(\numrange{15}{50} ans), il y a dissolution partielle du \ce{CO2}. Ce phénomène
est plus accentué dans le scénario optimal et il se reflète dans les traces
sommées, avec une arrivée anticipée pour le
réflecteur à \SI{700}{\milli\second} comparé au même réflecteur pour le suivi à
\num{15} ans (\cref{fig:seism_opt_c,fig:seism_bec_c}. Pour le scénario réaliste, on
peut observer une anomalie AVO à la base du réservoir. \\
Une comparaison entre le modèle hétérogène proposé et un modèle homogène
stratifié classique a été effectuée et les résultats sont présentés avec les
\cref{fig:stochvsblock_a,fig:stochvsblock_a} de l'article I à la
\cpageref{fig:stochvsblock_a} pour un suivi après \num{5} ans du début de
l’injection, pour le scénario réaliste. Cette comparaison permet d’évaluer les
différentes réponses sismiques produites par les deux modèles. Par exemple, le
réflecteur à \SI{700}{\milli\second} montre une diminution de l'amplitude avec
l'augmentation du déport pour le modèle homogène classique, tandis que pour le
modèle hétérogène le même réflecteur montre une augmentation de l'amplitude avec
le déport. Ce phénomène est confirmé avec l’analyse de Zoeprritz \citep{Aki1980}
dans la \cref{fig:zoeppritz} de l'article I à la \cpageref{fig:zoeppritz}.
