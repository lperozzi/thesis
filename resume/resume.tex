\chapter*{Résumé}
\addcontentsline{toc}{chapter}{Résumé} % Ajouter à la table des matières.
\setlength{\parindent}{0cm}
\setlength{\parskip}{1em}
Plusieurs rapports d'organismes internationaux tels que l'Agence internationale
de l'énergie (AIE) et le Groupe intergouvernemental d'experts sur l’évolution du
climat (GIEC) ont confirmé que le réchauffement climatique est indiscutable et
trouve son origine dans l'activité humaine. En même temps, d'ici \num{2040},
l'AIE prévoit une augmentation de la demande énergétique mondiale de
\SI{37}{\percent}. Bien que le choix des politiques et les évolutions du marché
devraient entraîner une baisse de la demande pour les combustibles fossiles,
ceci ne suffira pas à enrayer l'augmentation des émissions de dioxyde de
carbone, ce qui provoquera une accélération de la hausse de la température
mondiale
de \SI{3.6}{\degreeCelsius} à long terme. Le GIEC estime donc que pour limiter
cette hausse à \SI{2}{\degreeCelsius}, objectif adopté au niveau international
pour prévenir les répercussions les plus graves du changement climatique, le
monde ne devra pas émettre plus de \SI{1000}{\giga\tonne} \ce{CO2} à compter de
\num{2014}. \\
Selon les deux organismes, environ \SI{14}{\percent} des émissions seront
réduites grâce à l'emploi de la technologie du captage et stockage du \ce{CO2}.
Bien que cette technologie ne soit pas universellement reconnue parmi les
agences
de protection de l’environnement et les ONG, elle est la seule méthode à court
terme
qui permettrait d'avoir un impact significatif sur le bilan carbone et il y a
maintenant une urgence pour le déploiement du CSC au-delà de la phase
démonstrative.
Pour que le stockage géologique du \ce{CO2} ait un impact positif sur
l'environnement, le \ce{CO2} doit être stocké dans le sous-sol aussi longtemps
qu'il le faut pour que les émissions anthropologiques chutent à des niveaux
acceptables et dans des roches réservoirs permettant d'accueillir des volumes
importants de \ce{CO2}. Ces contraintes nécessitent que le \ce{CO2} soit stocké
sur
une
échelle de temps de l'ordre de \numrange{e1}{e4} ans. Pour atteindre cette
exigence, on doit s'assurer que le \ce{CO2} reste en place et ne puisse migrer
sur de grandes distances ni verticalement ni horizontalement. \par

À partir de ces constats, cette thèse propose une méthodologie de travail pour
la surveillance sismique temporelle et l’évaluation de l'incertitude liée à
l'injection du \ce{CO2} adaptée à un environnement avec des faibles porosités et
perméabilités, comme celui des Basses-Terres du St-Laurent (BTSL) au Québec,
Canada. Cette
méthodologie est menée sur deux fronts: utiliser les mesures de laboratoire et
la modélisation sismique de puits comme outils de haute résolution pour évaluer la
réponse sismique due à l'injection du \ce{CO2} et définir une séquence logique
de
modélisation stochastique d'un réservoir potentiel pour la séquestration
géologique du \ce{CO2}. Premièrement, les mesures de laboratoire sur deux
échantillons provenant des unités réservoir de BTSL ont permis d'évaluer la
réponse sismique due à l'injection du \ce{CO2} sous différentes conditions de
pression et température. Ces mesures ont permis de calibrer le modèle géologique
utilisé ensuite dans la modélisation sismique de puits. Cette modélisation a
montré que les différences rencontrées aux différents temps sont quantifiables
principalement par un délai de \SI{30}{\milli\second} associé à une diminution
des vitesses quand le \ce{CO2} supercritique remplace la saumure dans l’espace
poreux.\\
Ensuite, la modélisation numérique basée sur un modèle hétérogène réaliste de
l’aquifère salin des BSTL indique qu' à partir des données statiques initiales,
l'approche d'inversion stochastique par déformation graduelle permet
d'obtenir des estimations fidèles des propriétés physiques ainsi qu’une prédiction fiable
de la distribution du \ce{CO2} dans le réservoir.
